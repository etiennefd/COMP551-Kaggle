\documentclass[conference]{IEEEtran}
\usepackage{cite}
\usepackage{amsmath,amssymb,amsfonts}
\usepackage [utf8]{inputenc}
\usepackage{algorithmic}
\usepackage{graphicx}
\usepackage{textcomp}
\def\BibTeX{{\rm B\kern-.05em{\sc i\kern-.025em b}\kern-.08em
    T\kern-.1667em\lower.7ex\hbox{E}\kern-.125emX}}
\begin{document}

\title{Kaggle Competition: Find the ``Largest'' Digit\\
{\large COMP 551-001, March 2018\\}
{\Large Team Axolotl}
}

\author{\IEEEauthorblockN{Haohan Bo}
\IEEEauthorblockA{260782229 \\
haohan.bo@mail.mcgill.ca}
\and
\IEEEauthorblockN{Étienne Fortier-Dubois}
\IEEEauthorblockA{260430244 \\
etienne.fortier-dubois@mail.mcgill.ca}
\and
\IEEEauthorblockN{Martha Thomae Elias}
\IEEEauthorblockA{260658140 \\
martha.thomaeelias@mail.mcgill.ca}
}

\maketitle

\begin{abstract}
The main text of the report should not exceed 6 pages. References and appendix can be in excess of the 6 pages. The format should be double-column, 10pt font, min. 1” margins.
\end{abstract}

\section{Introduction}

Briefly describe the problem and summarize your approach and results.


\section{Feature Design}

Describe and justify your pre-processing methods, and how you designed and selected your features.

\subsection{Pre-processing}

Binarize images, isolate digits, find the largest one, remove the rest of the image, pad with black pixels ans resize. 

Isolate digits: one issue is that sometimes digits partially overlap. Separating these digits is a difficult problem. We have ignored it, because such cases are rare. 

To determine the largest digit, we found the one that had the largest dimension, for either of the width and height. This approach proved superior to using the area (width $\times$ height), as we found that images were classified according to the tallest or widest digit even though another digit may have had a larger total area. 



\section{Algorithms}

Give an overview of the learning algorithms used without going into too much detail in the class notes (e.g. SVM derivation, etc.), unless necessary to understand other details.

\subsection{SVM or logistic regression}

\subsection{Neural network}

\subsection{Third approach}


\section{Methodology}

Include any decisions about training/validation split, dis- tribution choice for naïve bayes, regularization strategy, any optimization tricks, setting hyper-parameters, etc.

\section{Results}

Present a detailed analysis of your results, including graphs and tables as appropriate. This analysis should be broader than just the Kaggle result: include a short comparison of the most important hyperparam- eters and all 3 methods you implemented.

% Commented out table and figure from the IEEE template: 

%\begin{table}[htbp]
%\caption{Table Type Styles}
%\begin{center}
%\begin{tabular}{|c|c|c|c|}
%\hline
%\textbf{Table}&\multicolumn{3}{|c|}{\textbf{Table Column Head}} \\
%\cline{2-4} 
%\textbf{Head} & \textbf{\textit{Table column subhead}}& \textbf{\textit{Subhead}}& \textbf{\textit{Subhead}} \\
%\hline
%copy& More table copy$^{\mathrm{a}}$& &  \\
%\hline
%\multicolumn{4}{l}{$^{\mathrm{a}}$Sample of a Table footnote.}
%\end{tabular}
%\label{tab1}
%\end{center}
%\end{table}

%\begin{figure}[htbp]
%\centerline{\includegraphics{fig1.png}}
%\caption{Example of a figure caption.}
%\label{fig}
%\end{figure}

\section{Discussion}
Discuss the pros/cons of your approach \& methodology and suggest areas of future work.


\section*{Statement of contributions}

Briefly describe the contributions of each team member towards each of the components of the project (e.g. defining the problem, developing the methodology, coding the solution, performing the data analysis, writing the report, etc.) At the end of the Statement of Contributions, add the following statement: 

We hereby state that all the work presented in this report is that of the authors.


\section*{References}

Optional.

% To cite: \cite{b1}.

%\begin{thebibliography}{00}
%\bibitem{b1} G. Eason, B. Noble, and I. N. Sneddon, ``On certain integrals of Lipschitz-Hankel type involving products of Bessel functions,'' Phil. Trans. Roy. Soc. London, vol. A247, pp. 529--551, April 1955.
%\bibitem{b2} J. Clerk Maxwell, A Treatise on Electricity and Magnetism, 3rd ed., vol. 2. Oxford: Clarendon, 1892, pp.68--73.
%\bibitem{b3} I. S. Jacobs and C. P. Bean, ``Fine particles, thin films and exchange anisotropy,'' in Magnetism, vol. III, G. T. Rado and H. Suhl, Eds. New York: Academic, 1963, pp. 271--350.
%\bibitem{b4} K. Elissa, ``Title of paper if known,'' unpublished.
%\bibitem{b5} R. Nicole, ``Title of paper with only first word capitalized,'' J. Name Stand. Abbrev., in press.
%\bibitem{b6} Y. Yorozu, M. Hirano, K. Oka, and Y. Tagawa, ``Electron spectroscopy studies on magneto-optical media and plastic substrate interface,'' IEEE Transl. J. Magn. Japan, vol. 2, pp. 740--741, August 1987 [Digests 9th Annual Conf. Magnetics Japan, p. 301, 1982].
%\bibitem{b7} M. Young, The Technical Writer's Handbook. Mill Valley, CA: University Science, 1989.
%\end{thebibliography}

\section*{Appendix}

Optional. Here you can include additional results, more detail of the methods, etc.


\end{document}
